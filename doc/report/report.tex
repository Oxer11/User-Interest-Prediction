% !TEX TS-program = xelatex
% !TEX encoding = UTF-8 Unicode
% !Mode:: "TeX:UTF-8"

%This file contains the LaTeX code of my laboratory report for my ICS II course.
%Author: 张作柏/Zuobai Zhang <17300240035@fudan.edu.cn>

% This is a simple template for a LaTeX document using the "article" class.
% See "book", "report", "letter" for other types of document.

\documentclass[12pt]{article} % use larger type; default would be 10pt

\usepackage[utf8]{inputenc} % set input encoding (not needed with XeLaTeX)

%%% Examples of Article customizations
% These packages are optional, depending whether you want the features they provide.
% See the LaTeX Companion or other references for full information.

%%% PAGE DIMENSIONS
\usepackage[top=1.05in, bottom=0.95in, left=0.90in, right=1.10in]{geometry}
%\usepackage{geometry} % to change the page dimensions
\geometry{a4paper} % or letterpaper (US) or a5paper or....
% \geometry{margin=2in} % for example, change the margins to 2 inches all round
% \geometry{landscape} % set up the page for landscape
%   read geometry.pdf for detailed page layout information

\usepackage{graphicx} % support the \includegraphics command and options

% \usepackage[parfill]{parskip} % Activate to begin paragraphs with an empty line rather than an indent

%%% PACKAGES
\usepackage{booktabs} % for much better looking tables
\usepackage{array} % for better arrays (eg matrices) in maths
\usepackage{paralist} % very flexible & customisable lists (eg. enumerate/itemize, etc.)
\usepackage{verbatim} % adds environment for commenting out blocks of text & for better verbatim
\usepackage{subfig} % make it possible to include more than one captioned figure/table in a single float
% These packages are all incorporated in the memoir class to one degree or another...

%%% HEADERS & FOOTERS
\usepackage{fancyhdr} % This should be set AFTER setting up the page geometry
\pagestyle{fancy} % options: empty , plain , fancy
%\renewcommand{\headrulewidth}{0pt} % customise the layout...
\lhead{}\chead{}\rhead{}
\lfoot{}\cfoot{\thepage}\rfoot{}

%%% SECTION TITLE APPEARANCE
\usepackage{sectsty}
\allsectionsfont{\sffamily\mdseries\upshape} % (See the fntguide.pdf for font help)
% (This matches ConTeXt defaults)

%%% ToC (table of contents) APPEARANCE
\usepackage[nottoc,notlof,notlot]{tocbibind} % Put the bibliography in the ToC
\usepackage[titles,subfigure]{tocloft} % Alter the style of the Table of Contents
\renewcommand{\cftsecfont}{\rmfamily\mdseries\upshape}
\renewcommand{\cftsecpagefont}{\rmfamily\mdseries\upshape} % No bold!
\usepackage{titletoc}
\titlecontents{section}
              [1.5cm]
              {\bf \large}%
              {\contentslabel{1.8em}}%
              {}%
              {\titlerule*[0.5pc]{$\cdot$}\contentspage\hspace*{0.6cm}}%
		   [\vspace{0.5em}]
\titlecontents{subsection}
              [1.8cm]
              {\normalsize}%
              {\contentslabel{2.0em}}%
              {}%
              {\titlerule*[0.5pc]{$\cdot$}\contentspage\hspace*{0.6cm}}%
		   [\vspace{0.4em}]
\titlecontents{subsubsection}
              [2.1cm]
              {\small}%
              {\contentslabel{2.5em}}%
              {}%
              {\titlerule*[0.5pc]{$\cdot$}\contentspage\hspace*{0.6cm}}%
		   [\vspace{0.4em}]


\usepackage[UTF8]{ctex}
\usepackage{fancyhdr}
\usepackage{enumerate}
\usepackage{indentfirst}
\usepackage{extramarks}
\usepackage{titling}
\usepackage{xcolor}
\usepackage{fontspec}
\usepackage[CJKbookmarks=true,colorlinks,linkcolor=black]{hyperref}
\setmainfont{Times New Roman}

%%% END Article customizations

%%% The "real" document content comes below...

%\title{\textbf{Digital Logic and Computer Design Report}}
\title{\textbf{KDD Cup 2012 Track 1解题报告}}
\author{张作柏\\17300240035}
%\date{} % Activate to display a given date or no date (if empty),
         % otherwise the current date is printed 

\begin{document}
\begin{sloppypar}
\maketitle

\pagestyle{fancy}
\lhead{\textbf{{\thetitle}}}
\rhead{\textbf{\nouppercase{\firstleftmark}}}
\cfoot{\thepage}

\thispagestyle{empty}
\tableofcontents
\clearpage

\setcounter{page}{1}


\section{任务简介}

本次PJ我选择了KDD Cup 2012 Track 1题目\footnote{\url{https://www.kaggle.com/c/kddcup2012-track1}},其中模型主要参考了~\cite{}一文。

\subsection{任务描述}

近年来,随着像Facebook、Twitter、腾讯微博等社交平台的发展,在线社交网络引起了广泛的关注。全中国最大的微博系统之一,腾讯微博,已经成为网络社交的重要平台。目前,腾讯微博拥有超过2亿的注册用户,每天产生四千万条信息。海量的数据引起了数据挖掘爱好者的注意,如何利用数据信息改善用户的使用体验,成为了一个十分有趣并值得研究的问题。

本任务中,我们需要根据用户的兴趣,预测他是否会关注某个对象(item)。对象可以是某个组织、个人、群体等等。最终我们要在所有备选推荐中,选择至多三个对象推荐给用户。

\subsection{数据信息}

\subsubsection{名词定义}

{\bf 对象(item):} 对象是腾讯微博中的一个用户,他可以代表组织、个人或群体。数据集中大约有六千个不同的对象。

{\bf 发微博(tweet):} 发微博是指用户可以在微博系统中发表一条信息,他的关注者会看到这条信息的提醒。

{\bf 转发(retweet):} 用户可以转发其他用户发表的信息,并在其下添加评论。

{\bf 评论(comment):} 用户可以在别人的微博下发表评论。

{\bf 关注者(follower):} 用户可以关注其他用户,若用户A关注了B,则称A是B的关注者。

\subsubsection{数据文件}

\begin{enumerate}
	\item {\bf 训练数据集rec\_log\_train.txt:} 记录了用户与对象之间的历史推荐结果。\\
	文件格式:(UserId) (ItemId) (Result) (Unix-timestamp)\\
	在Unix-timestamp的时间,系统向用户UserId推荐了物品ItemId,得到的结果为Result。Result为1,表示接受;Result为-1,表示拒绝。
	\item {\bf 测试数据集rec\_log\_test.txt:} 记录了测试集中用户与对象之间的可能推荐。\\
	文件格式同训练数据集rec\_log\_train.txt\\
	区别在于其中Result域为0,需要我们来预测。
	\item {\bf 用户信息user\_profile.txt:}
	\item {\bf 对象数据item.txt:}
	\item {\bf 用户行为user\_action.txt:}
	\item {\bf 用户关注行为user\_sns.txt:}
	\item {\bf 用户关键词描述user\_key\_word.txt:}
\end{enumerate}

\subsection{提交格式}

\subsection{评价方式}


\newpage
\section{数据预处理}

\subsection{数据清洗}

\subsection{成对训练}

\newpage
\section{用户兴趣模型}

\subsection{基本模型}

\subsection{年龄、性别因素}

\subsection{间接反馈信息}

\subsection{关键词、标签信息}

\subsection{用户、物品信息}


\newpage
\section{用户行为模型}

\subsection{时间度量}

\subsection{概率预测}

\newpage
\section{集成学习}

\subsection{模型原理}

\subsection{训练方法}


\newpage
\section{结果评估}

\subsection{测试模块}

\subsection{测试结果}

\newpage
\bibliographystyle{unsrt} 
\bibliography{PJ}

\end{sloppypar}
\end{document}
